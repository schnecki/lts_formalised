\documentclass[dvips,mnsc]{informs3}
% [citeauthoryear]

\OneAndAHalfSpacedXI % current default line spacing
%%\OneAndAHalfSpacedXII
%%\DoubleSpacedXII
%%\DoubleSpacedXI

% If hyperref is used, dvi-to-ps driver of choice must be declared as
%   an additional option to the \documentstyle. For example
%\documentclass[dvips,mnsc]{informs3}      % if dvips is used
%\documentclass[dvipsone,mnsc]{informs3}   % if dvipsone is used, etc.

% Private macros here (check that there is no clash with the style)

% --------------------

% \usepackage[table]{xcolor}
% \usepackage{xcolor}
% \usepackage[utf8x]{inputenc}
% \usepackage{listings}
% \usepackage{amsmath,amssymb}
% \usepackage[noend]{algpseudocode}
% \usepackage{stmaryrd}
% \usepackage{listings}
% \usepackage{algorithm}
% \usepackage{array}
% \usepackage{textcomp}
% \usepackage{pifont}
% \usepackage{bbding}
% \usepackage{comment}
% \usepackage{graphicx}
\usepackage{paper}
% \usepackage[svgpath=figures/]{svg}
% \usepackage[labelfont=bf]{caption}
% \captionsetup[table]{skip=5pt}

% \usepackage{academicons}
% \usepackage{cite}
% \usepackage[hidelinks,
% plainpages=false,
% pdftitle={
%   The Lead Time Syndrome
% },
% pdfauthor={Manuel Schneckenreither and Stefan Haeussler},
% pdfsubject={  The Lead Time Syndrome},
% pdfkeywords={production planning, order release, lead time syndrome, formalisation, detection,
%   operations research},
% ]{hyperref}

% \usepackage{url}

\newcommand\SH[2][r]{\ifx t#1 \textcolor{green!50!black}{[\textbf{SH:} #2]}
  \else \begin{center}\textcolor{green!50!black}{\textbf{SH:} #2} \end{center} \fi}

\newcommand\MS[2][r]{\ifx t#1 \textcolor{blue}{[\textbf{MS:} #2]}
  \else \begin{center}\textcolor{blue}{\textbf{MS:} #2} \end{center} \fi}

\newcolumntype{x}[1]{>{\centering\arraybackslash\hspace{0pt}}p{#1}}
% --------------------

% Natbib setup for author-year style
\usepackage{natbib}
 \bibpunct[, ]{(}{)}{,}{a}{}{,}%
 \def\bibfont{\small}%
 \def\bibsep{\smallskipamount}%
 \def\bibhang{24pt}%
 \def\newblock{\ }%
 \def\BIBand{and}%

%% Setup of theorem styles. Outcomment only one.
%% Preferred default is the first option.
\TheoremsNumberedThrough     % Preferred (Theorem 1, Lemma 1, Theorem 2)
%\TheoremsNumberedByChapter  % (Theorem 1.1, Lema 1.1, Theorem 1.2)
\ECRepeatTheorems

%% Setup of the equation numbering system. Outcomment only one.
%% Preferred default is the first option.
\EquationsNumberedThrough    % Default: (1), (2), ...
%\EquationsNumberedBySection % (1.1), (1.2), ...

% For new submissions, leave this number blank.
% For revisions, input the manuscript number assigned by the on-line
% system along with a suffix ".Rx" where x is the revision number.
\MANUSCRIPTNO{}

%%%%%%%%%%%%%%%%
\begin{document}
%%%%%%%%%%%%%%%%

% Outcomment only when entries are known. Otherwise leave as is and
%   default values will be used.
%\setcounter{page}{1}
%\VOLUME{00}%
%\NO{0}%
%\MONTH{Xxxxx}% (month or a similar seasonal id)
%\YEAR{0000}% e.g., 2005
%\FIRSTPAGE{000}%
%\LASTPAGE{000}%
%\SHORTYEAR{00}% shortened year (two-digit)
%\ISSUE{0000} %
%\LONGFIRSTPAGE{0001} %
%\DOI{10.1287/xxxx.0000.0000}%

% Author's names for the running heads
% Sample depending on the number of authors;
% \RUNAUTHOR{Jones}
% \RUNAUTHOR{Jones and Wilson}
% \RUNAUTHOR{Jones, Miller, and Wilson}
% \RUNAUTHOR{Jones et al.} % for four or more authors
% Enter authors following the given pattern:
\RUNAUTHOR{Schneckenreither and Haeussler}
%\RUNAUTHOR{}

% Title or shortened title suitable for running heads. Sample:
% \RUNTITLE{Bundling Information Goods of Decreasing Value}
% Enter the (shortened) title:
\RUNTITLE{The Lead Time Syndrome}

% Full title. Sample:
% \TITLE{Bundling Information Goods of Decreasing Value}
% Enter the full title:
\TITLE{A Formalisation of the Lead Time Syndrome}

% Block of authors and their affiliations starts here:
% NOTE: Authors with same affiliation, if the order of authors allows,
%   should be entered in ONE field, separated by a comma.
%   \EMAIL field can be repeated if more than one author
\ARTICLEAUTHORS{%
  \AUTHOR{Manuel Schneckenreither% \orcidID{0000-0002-4812-4665}
  }
  \AFF{Department of Information Systems, Production and Logistics Management, University of Innsbruck, Austria
  \EMAIL{manuel.schneckenreither@uibk.ac.at}} %, \URL{}}

  \AUTHOR{Stefan Haeussler % \orcidID{0000-0003-2589-1367}
  }
  \AFF{Department of Information Systems, Production and Logistics Management, University of Innsbruck, Austria
  \EMAIL{stefan.haeussler@uibk.ac.at}}
} % end of the block

\ABSTRACT{%
  TODO
% Enter your abstract
}%

% Sample
%\KEYWORDS{deterministic inventory theory; infinite linear programming duality;
%  existence of optimal policies; semi-Markov decision process; cyclic schedule}

% Fill in data. If unknown, outcomment the field
\KEYWORDS{production planning, order release, lead time syndrome, formalisation, operations
  research}
\HISTORY{ % This paper was first submitted on April 12, 1922 and has been with the authors for 83 years
  % for 65 revisions.
}

\maketitle
%%%%%%%%%%%%%%%%%%%%%%%%%%%%%%%%%%%%%%%%%%%%%%%%%%%%%%%%%%%%%%%%%%%%%%

% Samples of sectioning (and labeling) in MNSC
% NOTE: (1) \section and \subsection do NOT end with a period
%       (2) \subsubsection and lower need end punctuation
%       (3) capitalization is as shown (title style).
%
%\section{Introduction.}\label{intro} %%1.
%\subsection{Duality and the Classical EOQ Problem.}\label{class-EOQ} %% 1.1.
%\subsection{Outline.}\label{outline1} %% 1.2.
%\subsubsection{Cyclic Schedules for the General Deterministic SMDP.}
%  \label{cyclic-schedules} %% 1.2.1
%\section{Problem Description.}\label{problemdescription} %% 2.

% Text of your paper here

\section{Introduction}
\label{sec:introduction}

Lead Time Syndrome (LTS)


\section{The Lead Time Syndrome}
\label{sec:The_Lead_Time_Syndrome}

In this section we present a formalisation of the \LTS{}. It is straightforward to implement the
presented methods to establish a \LTS{} detection system. We follow the notation established
by~\cite{selccuk2013adaptive}.

\subsection{Single-Stage Manufacturing System}
\label{subsec:Single-Stage_Manufacturing_System}


Consider a single product one-stage manufacturing system that releases orders from the order pool
into the production system by an external mechanism. We denote the demand rate at time
$t \geqslant 0$ as $\dr{t}$, and the demand for the item during the period defined by timestamps
$t_{2} \geqslant t_{1} \geqslant 0$ as $\di{t_{1}}{t_{2}}$. Similarly we denote the order release
rate as $\rr{t}$ and the releases within a period as $\ri{t_{1}}{t_{2}}$. Usually

\begin{definition}

  The overall queue length

\end{definition}


Zumindest durschnittlich gleichbleibende Freigabemenge bei über die Zeit steigender EDT (oder
eventuell die Warteschlangenlänge [von Stage 1]).


\begin{definition}
  The \emph{Lead Time Syndrome} is defined as
\end{definition}

% *Informelle Definition LTS_+*


\subsection{Multi-Stage Manufacturing System}
\label{subsec:Multi-Stage_Manufacturing_System}


We now lift the established metric to multi-stage manufacturing systems.


\section{Experimental Evidence}
\label{sec:Experimental_Evidence}


\section{Conclusion}
\label{sec:Conclusion}


\bibliography{references}
\bibliographystyle{plainnat}

\end{document}

% --------------------
%%% Local Variables:
%%% mode: latex
%%% TeX-master: t
%%% End:

\documentclass[mnsc]{informs3}
% [citeauthoryear]

\OneAndAHalfSpacedXI % current default line spacing
%%\OneAndAHalfSpacedXII
%%\DoubleSpacedXII
%%\DoubleSpacedXI

% If hyperref is used, dvi-to-ps driver of choice must be declared as
%   an additional option to the \documentstyle. For example
%\documentclass[dvips,mnsc]{informs3}      % if dvips is used
%\documentclass[dvipsone,mnsc]{informs3}   % if dvipsone is used, etc.

% Private macros here (check that there is no clash with the style)

% --------------------
\usepackage[utf8x]{inputenc}
% \usepackage[table]{xcolor}
% \usepackage{xcolor}
% \usepackage{listings}
% \usepackage{amsmath,amssymb}
% \usepackage[noend]{algpseudocode}
% \usepackage{stmaryrd}
% \usepackage{listings}
% \usepackage{algorithm}
% \usepackage{array}
% \usepackage{textcomp}
% \usepackage{pifont}
% \usepackage{bbding}
% \usepackage{comment}
% \usepackage{graphicx}
\usepackage{paper}
% \usepackage[svgpath=figures/]{svg}
% \usepackage[labelfont=bf]{caption}
% \captionsetup[table]{skip=5pt}

% \usepackage{academicons}
% \usepackage{cite}
% \usepackage[hidelinks,
% plainpages=false,
% pdftitle={
%   The Lead Time Syndrome
% },
% pdfauthor={Manuel Schneckenreither and Stefan Haeussler},
% pdfsubject={  The Lead Time Syndrome},
% pdfkeywords={production planning, order release, lead time syndrome, formalisation, detection,
%   operations research},
% ]{hyperref}

% \usepackage{url}


\newcommand\SH[2][r]{\ifx t#1 \textcolor{green!50!black}{[\textbf{SH:} #2]}
  \else \begin{center}\textcolor{green!50!black}{\textbf{SH:} #2} \end{center} \fi}

\newcommand\MS[2][r]{\ifx t#1 \textcolor{blue}{[\textbf{MS:} #2]}
  \else \begin{center}\textcolor{blue}{\textbf{MS:} #2} \end{center} \fi}

\newcolumntype{x}[1]{>{\centering\arraybackslash\hspace{0pt}}p{#1}}
% --------------------

% Natbib setup for author-year style
\usepackage{natbib}
 \bibpunct[, ]{(}{)}{,}{a}{}{,}%
 \def\bibfont{\small}%
 \def\bibsep{\smallskipamount}%
 \def\bibhang{24pt}%
 \def\newblock{\ }%
 \def\BIBand{and}%

%% Setup of theorem styles. Outcomment only one.
%% Preferred default is the first option.
\TheoremsNumberedThrough     % Preferred (Theorem 1, Lemma 1, Theorem 2)
%\TheoremsNumberedByChapter  % (Theorem 1.1, Lema 1.1, Theorem 1.2)
\ECRepeatTheorems

%% Setup of the equation numbering system. Outcomment only one.
%% Preferred default is the first option.
\EquationsNumberedThrough    % Default: (1), (2), ...
%\EquationsNumberedBySection % (1.1), (1.2), ...

% For new submissions, leave this number blank.
% For revisions, input the manuscript number assigned by the on-line
% system along with a suffix ".Rx" where x is the revision number.
\MANUSCRIPTNO{}

%%%%%%%%%%%%%%%%
\begin{document}
%%%%%%%%%%%%%%%%

% Outcomment only when entries are known. Otherwise leave as is and
%   default values will be used.
%\setcounter{page}{1}
%\VOLUME{00}%
%\NO{0}%
%\MONTH{Xxxxx}% (month or a similar seasonal id)
%\YEAR{0000}% e.g., 2005
%\FIRSTPAGE{000}%
%\LASTPAGE{000}%
%\SHORTYEAR{00}% shortened year (two-digit)
%\ISSUE{0000} %
%\LONGFIRSTPAGE{0001} %
%\DOI{10.1287/xxxx.0000.0000}%

% Author's names for the running heads
% Sample depending on the number of authors;
% \RUNAUTHOR{Jones}
% \RUNAUTHOR{Jones and Wilson}
% \RUNAUTHOR{Jones, Miller, and Wilson}
% \RUNAUTHOR{Jones et al.} % for four or more authors
% Enter authors following the given pattern:
\RUNAUTHOR{Schneckenreither and Haeussler}
%\RUNAUTHOR{}

% Title or shortened title suitable for running heads. Sample:
% \RUNTITLE{Bundling Information Goods of Decreasing Value}
% Enter the (shortened) title:
\RUNTITLE{The Lead Time Syndrome}

% Full title. Sample:
% \TITLE{Bundling Information Goods of Decreasing Value}
% Enter the full title:
\TITLE{A Formalisation of the Lead Time Syndrome}

% Block of authors and their affiliations starts here:
% NOTE: Authors with same affiliation, if the order of authors allows,
%   should be entered in ONE field, separated by a comma.
%   \EMAIL field can be repeated if more than one author
\ARTICLEAUTHORS{%
  \AUTHOR{Manuel Schneckenreither% \orcidID{0000-0002-4812-4665}
  }
  \AFF{Department of Information Systems, Production and Logistics Management, University of Innsbruck, Austria
  \EMAIL{manuel.schneckenreither@uibk.ac.at}} %, \URL{}}

  \AUTHOR{Stefan Haeussler % \orcidID{0000-0003-2589-1367}
  }
  \AFF{Department of Information Systems, Production and Logistics Management, University of Innsbruck, Austria
  \EMAIL{stefan.haeussler@uibk.ac.at}}
} % end of the block

\ABSTRACT{%
  TODO
% Enter your abstract
}%

% Sample
%\KEYWORDS{deterministic inventory theory; infinite linear programming duality;
%  existence of optimal policies; semi-Markov decision process; cyclic schedule}

% Fill in data. If unknown, outcomment the field
\KEYWORDS{production planning, order release, lead time syndrome, formalisation, operations
  research}
\HISTORY{ % This paper was first submitted on April 12, 1922 and has been with the authors for 83 years
  % for 65 revisions.
}

\maketitle
%%%%%%%%%%%%%%%%%%%%%%%%%%%%%%%%%%%%%%%%%%%%%%%%%%%%%%%%%%%%%%%%%%%%%%

% Samples of sectioning (and labeling) in MNSC
% NOTE: (1) \section and \subsection do NOT end with a period
%       (2) \subsubsection and lower need end punctuation
%       (3) capitalization is as shown (title style).
%
%\section{Introduction.}\label{intro} %%1.
%\subsection{Duality and the Classical EOQ Problem.}\label{class-EOQ} %% 1.1.
%\subsection{Outline.}\label{outline1} %% 1.2.
%\subsubsection{Cyclic Schedules for the General Deterministic SMDP.}
%  \label{cyclic-schedules} %% 1.2.1
%\section{Problem Description.}\label{problemdescription} %% 2.

% Text of your paper here

\section{Introduction}
\label{sec:introduction}

Introduce abbreviations:
\begin{itemize}
\item Lead Time Syndrome (\LTS{})
\item Work-in-process (\WIP{})
\end{itemize}


Related literature \MS[t]{Cite!} argues that the lead time syndrome can only occur (i) if a lead
time is set dynamically and nonreciprocal according to the current \WIP{}-level and (ii) the order
releases monotonically depend on this lead time. In a practical point of view this assumption is
correct, however theoretically the interconnection between \WIP{}-level and number of releases is
not needed to infer this systematic releasing misbehavior.

\begin{proof}{Proof}
  Consider a manufacturer with constant demand and an order release mechanism which strictly
  increases the number of orders to be released on every release decision, e.g.~by 1 order. Clearly
  the release decision does neither depend on the flow time nor lead time. However, considered over
  time this leads to a monotonically increasing \WIP{}-level which again leads to a monotonically
  increasing lead time.\qed
\end{proof}

% \MS{name clash: LEAD TIME Syndrome with no LEAD TIME?}
Despite this theoretical observation the name lead time syndrome may be under question. However, as
in cases like the one given in the proof above the lead time follows the order release decisions and
in practice order releases are usually determined by the lead time (or flow time), the wording can
be considered as appropriate.
%
Nonetheless, our method is based on this observation. First we do not consider the lead time as
input variable to the method, then we do not suppose an interconnection between the \WIP{}-level
with the release decision procedure, and finally we do not restrict our method on any particular
order release pattern.


\section{The Lead Time Syndrome}
\label{sec:The_Lead_Time_Syndrome}

In this section we present a formalisation of the \LTS{}.
%
\MS[t]{TODO} The section is split into two
parts. First we will present the method for single-stage manufacturers, before considering more
stages.
%
It is straightforward to implement the presented methods to establish a \LTS{} detection system, see
Section~\ref{sec:Experimental_Evidence}. We follow the notation established
by~\cite{selccuk2013adaptive} to ease readability.

\subsection{Single-Stage Manufacturing System}
\label{subsec:Single-Stage_Manufacturing_System}


Consider a single product one-stage flow-shop manufacturing system that releases orders from the
order pool into the production system by an external mechanism.
%
Usually the orders are released based on the observed (average) flow time of the last finished
orders. However, our method does not presuppose such an interdependence for detecting the lead time
syndrome as it works directly on (parts of) the \WIP{}-level of the system. Thus our approach is
also able to observe systematic releasing misbehavior, referred to as \LTS{}, which do not consider
the lead time as depended variable for the release decisions.
%
% Intuitively the method works as
% follows. By measuring the change of the average \WIP{}-level over time and comparing this to the
% average releases into the system it can infer the desired property.
%
We denote the demand rate at time $t \geqslant 0$ as $\dr{t}$, and the demand for the item during
the period defined by timestamps $t_{2} \geqslant t_{1} \geqslant 0$ as $\di{t_{1}}{t_{2}}$.
Similarly we denote the order release rate as $\rr{t}$ and the releases within a period as
$\ri{t_{1}}{t_{2}}$. Each work-station $i$ is equipped with a queue $\q_{i}$ on which released
orders are appended according to a given routing and from there handed over to the work-station if
only if the work-station is empty and the order is at the head of the queue. Thus we allow parallel
processing of unique orders with non-shared queues in the setup. In case of shared queues in this
single item setup the work-station shall be considered as a single work-station. The size of a queue
$\size{\q_{i}} \geqslant 0$ is given by the number of orders within the queue.


\begin{definition}

  The queue size $\size{\Q_{i}}$ of stage $i$ is defined as
  $\size{\Q_{i}} = \sum_{i=1}^{n} \size{\q_{i}}$, where $n$ is the number of work-stations of stage
  $i$. Note that in this subsection we set $i=1$.

\end{definition}


As the demand may vary over time, e.g.~in a seasonal pattern or simply by some variance, it must be
considered as time factor in the formulation of the lead time syndrome. To clarify, consider a
seasonal demand. It can clearly not be considered a lead time syndrome if the releases just depict
the demand.

Let the periodic demand be given by $\di{t}{t+1} = \dmean(\frac{t+t+1}{2}) + \epsilon$,
where $\dmean$ is a function which maps any given time $t \geqslant 0$ to the expected mean demand
at time $t$ and $\epsilon$ is drawn from a random variable $\X$ with standard deviation $\sigma$.

\begin{example}
  In case of a constant demand, $\dmean$ maps to a constant value, whereas in a seasonal demand $f$
  might be the sine-function. In most cases we expect the random variable to be drawn from the
  standard distribution: $\X \sim \N{0}{\sigma^{2}}$.
\end{example}


- je mehr variation im demand desto länger die zeit


and therefore specifies the


Zumindest durschnittlich gleichbleibende Freigabemenge bei über die Zeit steigender EDT (oder
eventuell die Warteschlangenlänge [von Stage 1]).


\begin{definition}
  The \emph{Lead Time Syndrome} is defined as
\end{definition}

% *Informelle Definition LTS_+*


\subsection{Multi-Stage Manufacturing System}
\label{subsec:Multi-Stage_Manufacturing_System}


We now lift the established metric to multi-stage manufacturing systems.


\section{Experimental Evidence}
\label{sec:Experimental_Evidence}


\section{Conclusion}
\label{sec:Conclusion}


\bibliography{references}
\bibliographystyle{plainnat}

\end{document}

% --------------------
%%% Local Variables:
%%% mode: latex
%%% TeX-master: t
%%% End:
